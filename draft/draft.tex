\documentclass[a4paper,10pt]{article}

% packages
\usepackage[utf8]{inputenc} % allows usage of spanish special characters
\usepackage[spanish,english]{babel} % english dictionary for proper hyphenation
\usepackage{amsmath} % math expressions
\usepackage{upgreek} % upright greek letters for math
\usepackage{txfonts} % nice fonts 
\usepackage{authblk} % author affiliations
\usepackage{graphicx} % images
\usepackage{float} % image positioning
\usepackage{multirow} % allows merging cells in tables
\usepackage{makecell} % more table customization
\usepackage{mathtools} % nice matrices
\usepackage{rotating} % rotated text
\usepackage{caption} % customization of figure/table captions
\usepackage{subcaption} % collages of multiple images
\usepackage{hyperref} % hyperlinks
\usepackage{nameref} % cross-references between draft and supplementary material
\usepackage{zref-xr,zref-user} % more cross-references formatting
\usepackage[square,numbers,sort&compress]{natbib}
\usepackage[table,xcdraw]{xcolor} % text and table background colors
\usepackage[margin=3cm]{geometry} % margins
\usepackage[nottoc,numbib]{tocbibind} % add references to table of contents
\usepackage{lineno} % line numbers

\usepackage{pdflscape} %% Landscape
\usepackage{dcolumn}
\usepackage{booktabs}
\usepackage{savesym}
\savesymbol{iint}
\restoresymbol{TXF}{iint}
%% \usepackage{pdfpages} %% to include a pdf file

% customization
\setcounter{Maxaffil}{0} % affiliations
\renewcommand\Affilfont{\itshape\small} % style of affiliations text
\makeatletter\renewcommand{\@biblabel}[1]{#1.}\makeatother % change [number] for number. in reference list
\addto{\captionsenglish}{\renewcommand{\bibname}{References}}
\newcommand{\idea}[1]{\textcolor{red}{#1}}
\newcommand{\mr}{\textcolor{red}{\textbf{[missing ref(s)]}}}
\newcommand{\Burl}[1]{\url{#1}}
\zxrsetup{toltxlabel} % cross references: draft & suppl mat
\zexternaldocument*[M-]{draft}
\hypersetup{
  colorlinks = true,
  citecolor=  black,
  linkcolor = {blue},
  filecolor = cyan % controls color of external ref, if used
}
\captionsetup[figure]{font=small, labelfont={bf}, labelformat={default},
   labelsep=period, name={Fig.}} % custom figure captions
\captionsetup[table]{font=small, labelfont={bf}, labelformat={default},
   labelsep=period, name={Table}} % custom table captions
\hypersetup{citecolor={blue}}
\definecolor{lightgray}{gray}{0.9}

% title, authors, affiliations
\title{Provisional Title}
\author[1,2,3]{Authors}
\author[1,2,$\dagger$]{Álvaro Sánchez}
\affil[1]{Department of Ecology \& Evolutionary Biology,
Yale University, New Haven, CT, USA}
\affil[2]{Microbial Sciences Institute,
Yale University, New Haven, CT, USA}
\affil[3]{Other affiliations...}
\affil[$\dagger$]{To whom correspondence should be addressed: \normalfont alvaro.sanchez@yale.edu}
\date{}

\newcommand\legendtreefigs{Internal nodes show the partitioning variables   and the values that lead to each partition; light blue denotes the   values or categories that lead down the left split and salmon the values   or categories that lead down the right split (partitioning criteria are   also shown in the edges leading down a node). Leaves or terminal nodes   show (weighted) boxplots of the dependent variable for the observations   that belong that fall in that leave and, as label, the name of the node   ax_3,nd the sample size (sum of weighted observations and \% of the total   data set for the fit). The dot plots underneath the boxplots show the mean JS for all the methods for that leave.}   



  
\begin{document}

\linenumbers

\maketitle

\begin{abstract}
  
The abstract goes here.
  
\end{abstract}

\section{Introduction}\label{intro}

This is an example cite \cite{Vetrovsky2013,nloptr}

\section{Results}\label{results}

\section{Discussion}\label{discussion}

\section{Methods}\label{methods}

\subsection{Simulations}\label{methods:sim}

We used the Community Simulator package \cite{Marsland2020} and included new
features for our simulations. In the package,
species are characterized by their resource uptake rates ($c_{i\alpha}$ for
species $i$ and resource $\alpha$), and they all
share a common metabolic matrix $\mathbf{D}$.
The element $D_{\alpha\beta}$
of this matrix represents the fraction of energy in the form of resource $\alpha$
secreted when resource $\beta$ is consumed.
Here we implemented a new operation mode
in which species can secrete different metabolites (and/or
in different abundances) when consuming a same resource. Experimental observations
support the idea of distinct species producing different sets of byproducts when
growing in the same primary resource \mr. We let $D_{i\alpha\beta}$ denote the
fraction of energy in the form of resource $\alpha$ secreted \textit{by species
i} when consuming resource $\beta$ ---note that now $D_{i\alpha\beta}$ need not be
equal to $D_{j\alpha\beta}$ if $i \neq j$, unlike in the original Community
Simulator. In the package's underlying Microbial Consumer Resource Model
\cite{Goldford2018,Marsland2019}, this just means that the energy flux
$J^{out}_{i\beta}$ now takes the form

\begin{equation}
J^{out}_{i\beta} = \sum_\alpha D_{i\beta\alpha} l_\alpha J^{in}_{i_\alpha}
\label{eq:jout}
\end{equation}
%
The documentation for the Community Simulator package contains a detailed
description of the model, parameters and package use. For the updated package with
this new functionality, see \nameref{datacode}.

For the simulations here,
we first generate a library of 660 species (divided into three specialist
families of 200 memebers each
and a generalist family of 50 members)
and 30 resources (divided into three classes of 10 members each).
We randomly sample 50 species from each pool in equal ratios to seed
100 resident and
100 invasive communities respectively.
We then grow and dilute the communities serially,
replenishing the primary
resource after each dilution.
We repeat the process 20 times to ensure generational equilibrium is
achieved.

\section{Data \& code availability}\label{datacode}

Experimental data and code for the analysis, as well as code for the simulations
and the updated Community Simulator package with instructions to use the
new features can be found in \url{github.com/jdiazc9/coalescence}.

\section{Figures}\label{figs}

% references
\clearpage
\bibliographystyle{mystyle}
%\bibliographystyle{unsrt}
\bibliography{refs}

\end{document}




















