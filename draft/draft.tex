\documentclass[a4paper,10pt]{article}

% packages
\usepackage[utf8]{inputenc} % allows usage of spanish special characters
\usepackage[spanish,english]{babel} % english dictionary for proper hyphenation
\usepackage{amsmath} % math expressions
\usepackage{upgreek} % upright greek letters for math
\usepackage{txfonts} % nice fonts 
\usepackage{authblk} % author affiliations
\usepackage{graphicx} % images
\usepackage{float} % image positioning
\usepackage{multirow} % allows merging cells in tables
\usepackage{makecell} % more table customization
\usepackage{mathtools} % nice matrices
\usepackage{rotating} % rotated text
\usepackage{caption} % customization of figure/table captions
\usepackage{subcaption} % collages of multiple images
\usepackage{hyperref} % hyperlinks
\usepackage{nameref} % cross-references between draft and supplementary material
\usepackage{zref-xr,zref-user} % more cross-references formatting
\usepackage[square,numbers,sort&compress]{natbib}
\usepackage[table,xcdraw]{xcolor} % text and table background colors
\usepackage[margin=3cm]{geometry} % margins
\usepackage[nottoc,numbib]{tocbibind} % add references to table of contents
\usepackage{lineno} % line numbers

\usepackage{pdflscape} %% Landscape
\usepackage{dcolumn}
\usepackage{booktabs}
\usepackage{savesym}
\savesymbol{iint}
\restoresymbol{TXF}{iint}
%% \usepackage{pdfpages} %% to include a pdf file

% customization
\setcounter{Maxaffil}{0} % affiliations
\renewcommand\Affilfont{\itshape\small} % style of affiliations text
\makeatletter\renewcommand{\@biblabel}[1]{#1.}\makeatother % change [number] for number. in reference list
\addto{\captionsenglish}{\renewcommand{\bibname}{References}}
\newcommand{\idea}[1]{\textcolor{red}{#1}}
\newcommand{\mr}{\textcolor{red}{\textbf{[missing ref(s)]}}}
\newcommand{\Burl}[1]{\url{#1}}
\zxrsetup{toltxlabel} % cross references: draft & suppl mat
\zexternaldocument*[M-]{draft}
\hypersetup{
  colorlinks = true,
  citecolor=  black,
  linkcolor = {blue},
  filecolor = cyan % controls color of external ref, if used
}
\captionsetup[figure]{font=small, labelfont={bf}, labelformat={default},
   labelsep=period, name={Fig.}} % custom figure captions
\captionsetup[table]{font=small, labelfont={bf}, labelformat={default},
   labelsep=period, name={Table}} % custom table captions
\hypersetup{citecolor={blue}}
\definecolor{lightgray}{gray}{0.9}

% title, authors, affiliations
\title{Provisional Title}
\author[1,2,3]{Authors}
\author[1,2,$\dagger$]{Álvaro Sánchez}
\affil[1]{Department of Ecology \& Evolutionary Biology,
Yale University, New Haven, CT, USA}
\affil[2]{Microbial Sciences Institute,
Yale University, New Haven, CT, USA}
\affil[3]{Other affiliations...}
\affil[$\dagger$]{To whom correspondence should be addressed: \normalfont alvaro.sanchez@yale.edu}
\date{}



  
\begin{document}

\linenumbers

\maketitle

\begin{abstract}
  
The abstract goes here.
  
\end{abstract}

\section{Introduction}\label{intro}

This is an example cite \cite{Vetrovsky2013,nloptr}

\section{Results}\label{results}

\section{Discussion}\label{discussion}

\section{Methods}\label{methods}

\subsection{Stabilization of environmental communities in simple synthetic environments}
\label{methods:community-assembly}

Communities were stabilized \textit{ex situ} as described in \cite{Goldford2018}.
In short, environmental samples (soil, leaves...) within one meter radius in eight different
geographical locations were collected with sterile
tweezers or spatulas into 50mL sterile tubes (Fig. \mr).
One gram of each sample was allowed to
sit at room temperature in 10mL of phosphate buffered saline (1$\times$PBS) containing
200$\upmu$g/mL cycloheximide to suppress eukaryotic growth.
After 48h, samples were mixed 1:1 with 80\% glycerol and kept frozen at -80$^\circ$C.
Starting microbial communities were prepared by scrapping the frozen stocks into
200$\upmu$L of 1$\times$PBS and adding a volume of 4$\upmu$L to 500$\upmu$L
of synthetic minimal media (1$\times$M9) supplemented with 200$\upmu$g/mL cycloheximide
and 0.07 C-mol/L glutamine or sodium citrate as the carbon source in 96 deep-well plates
(1.2mL; VWR).
Cultures were then incubated still at 30$^\circ$C to allow for re-growth.
After 48h, samples were fully homogenized and biomass increase was followed by measuring
the optical density (620nm) of 100$\upmu$L of the cultures in a Multiskan FC plate reader
(Thermo Scientific).
Communities were stabilized \cite{Goldford2018} by passaging 4$\upmu$L of the cultures into
500$\upmu$L of fresh media (1$\times$M9 with the carbon source)  every 48h for a total of
12 transfers at a dilution factor of 1:100,
roughly equivalent to 80 generations per culture (Fig. \mr).
Cycloheximide was not added to the media after the first two transfers.

\subsection{Isolation of dominant species}\label{methods:dominants}

For each community, the most abundant colony morphotype at the end of the ninth transfer
was selected, resuspended in 100$\upmu$L 1$\times$PBS and serially diluted (1:10).
Next, 20$\upmu$L of the cells diluted to $10^{-6}$ were plated in the corresponding synthetic
minimal media and allowed to regrow at 30$^\circ$C for 48h. Dominants were then inoculated
into 500$\upmu$L of fresh media and incubated still at 30$^\circ$C for 48h.
After this period, the communities stabilized for eleven transfers and the isolated dominants
were ready for the competition experiments (Fig \mr) at the onset of the twelfth transfer.

\subsection{Simulations}\label{methods:sim}

We used the Community Simulator package \cite{Marsland2020} and included new
features for our simulations. In the package,
species are characterized by their resource uptake rates ($c_{i\alpha}$ for
species $i$ and resource $\alpha$), and they all
share a common metabolic matrix $\mathbf{D}$.
The element $D_{\alpha\beta}$
of this matrix represents the fraction of energy in the form of resource $\alpha$
secreted when resource $\beta$ is consumed.
Here we implemented a new operation mode
in which species can secrete different metabolites (and/or
in different abundances) when consuming a same resource. Experimental observations
support the idea of distinct species producing different sets of byproducts when
growing in the same primary resource \mr. We call $D_{i\alpha\beta}$ to the
fraction of energy in the form of resource $\alpha$ secreted \textit{by species
i} when consuming resource $\beta$ ---note that now $D_{i\alpha\beta}$ need not be
equal to $D_{j\alpha\beta}$ if $i \neq j$, unlike in the original Community
Simulator. In the package's underlying Microbial Consumer Resource Model
\cite{Goldford2018,Marsland2019}, this just means that the energy flux
$J^{\mathrm{out}}_{i\beta}$ now takes the form

\begin{equation}
J^{\mathrm{out}}_{i\beta} = \sum_\alpha D_{i\beta\alpha} l_\alpha J^{\mathrm{in}}_{i\alpha}
\label{eq:jout}
\end{equation}
%
The documentation for the Community Simulator contains detailed
descriptions of the model, parameters and package use. For the updated package with
the new functionality, see \nameref{datacode}.

For our simulations,
we first generate a library of 660 species (divided into three specialist
families of 200 members each
and a generalist family of 60 members)
and 30 resources (divided into three classes of 10 members each).
We split this library into two non-overlapping pools of 330 species each.
We randomly sample 50 species from each pool in equal ratios to seed
100 resident and
100 invasive communities respectively.
We then grow and dilute the communities serially,
replenishing the primary
resource after each dilution.
We repeat the process 20 times to ensure generational equilibrium is
achieved \cite{Goldford2018}.
We then perform the \textit{in silico} experiments by using the
generationally stable communities to seed 100 coalesced communities
that we again stabilize as described previously.
Similarly, we identify the dominant (most
abundant) species of every resident and invasive community to carry out pairwise
competition and single invasion simulations.
Most parameters are set to the defaults of the original Community Simulator
package. Table \mr shows those that are given non-default values to ensure
enough variation in the primary communities.

\section{Data \& code availability}\label{datacode}

Experimental data and code for the analysis, as well as code for the simulations
and the updated Community Simulator package with instructions to use the
new features can be found in \url{github.com/jdiazc9/coalescence}.

\section{Figures}\label{figs}

% references
\clearpage
\bibliographystyle{mystyle}
%\bibliographystyle{unsrt}
\bibliography{refs}

\end{document}




















